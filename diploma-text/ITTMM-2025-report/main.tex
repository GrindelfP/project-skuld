%%
\documentclass[%
]{ittmm}


% \usepackage[T1]{fontenc}% T2A for Cyrillic font encoding
% \usepackage[english, russian]{babel}

%%% One can fix some overfulls
\sloppy

%% Minted listings support
%% Need pygment <http://pygments.org/> <http://pypi.python.org/pypi/Pygments>
\usepackage{minted}
%% auto break lines
\setminted{breaklines=true}

%% end of the preamble, start of the body of the document source.
\begin{document}

%%
%% Rights management information.
%% CC-BY is default license.
\copyrightyear{2025}
\copyrightclause{Copyright for this paper by its authors.
  Use permitted under Creative Commons License Attribution 4.0
  International (CC BY 4.0).}

%%
%% This command is for the conference information
\conference{Information and Telecommunication Technologies and Mathematical Modeling of High-Tech Systems 2025 (ITTMM 2025), Moscow, April 07--11, 2025}

%%
%% The "title" command
\title{Application of Neural Network Approach for Numerical Integration}

% \tnotemark[1]
% \tnotetext[1]{You can use this document as the template for preparing your publication. We recommend using the latest version of the ittmm style.}

%%
%% The "author" command and its associated commands are used to define
%% the authors and their affiliations.
\author[1]{Gregory Alexandrovich Shipunov}[%
orcid=0009-0007-7819-641X,
email=shgregory3@gmail.com,
]
\cormark[1]

\author[2]{Oksana Ivanovna Streltsova}[%
orcid=!!!,
email=!!!,
]

\author[2]{Yuriy Leonidovich Kalinovskiy}[%
orcid=!!!,
email=!!!,
]
\address[1]{Dubna State University,
  19 Universitetskaya St, Dubna, 141980, Russian Federation}
\address[2]{Joint Institute for Nuclear Research,
  6 Joliot-Curie St, Dubna, 141980, Russian Federation}

%%% Footnotes
\cortext[1]{Corresponding author.}

%%
%% The abstract is a short summary of the work to be presented in the
%% article.
\begin{abstract}
    The short abstract should have between 150 and 250 words.
    Papers should contain 3-6 full pages, including figures, tables, and references.
    When a paper exceeds 6 pages, extra pages can be chargeable.
    The materials are made out according to the template of the conference book and are laid out on the Conference website.
    The title and the authors of the extended abstract should be presented using the proper text formatting like in the template above.
\end{abstract}

%%
%% Keywords. The author(s) should pick words that accurately describe
%% the work being presented. Separate the keywords with commas.
\begin{keywords}
  neural networks \sep
  numerical integration \sep
  meson
\end{keywords}

%% This command processes the author and affiliation and title
%% information and builds the first part of the formatted document.
\maketitle

\section{Introduction}

introduction text \cite{lloyd2020using}

\section{Neural Network Approach}

NNI theory \cite{costa2003pseudoscalar} \cite{blaschke2012meson}

\section{Physics task}

text

\section{Usage of Neural Network Approach in the Physics task}

text

\section{Future development}

text


%%
%% Define the bibliography file to be used
\bibliography{main}


\end{document}

%%% Local Variables:
%%% mode: LaTeX
%%% TeX-master: t
%%% End:
